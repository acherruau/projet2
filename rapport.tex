\documentclass[a4paper,12pt]{article}
\usepackage[utf8]{inputenc}
\usepackage[francais]{babel}
\usepackage[T1]{fontenc}
\usepackage[pdftex]{graphicx}
\usepackage{url}

\usepackage{fancyhdr}
\pagestyle{fancy}






\setlength{\parindent}{0cm}
\setlength{\parskip}{1ex plus 0.5ex minus 0.2ex}
\newcommand{\hsp}{\hspace{20pt}}
\newcommand{\HRule}{\rule{\linewidth}{0.5mm}}
%opening

\renewcommand{\headrulewidth}{1pt}
\fancyhead[L]{\leftmark}
\fancyhead[R]{LaTeX}

\renewcommand{\footrulewidth}{1pt}
\fancyfoot[C]{\textbf{page \thepage}} 




\begin{document}

\begin{titlepage}
  \begin{sffamily}
  \begin{center}

    % Upper part of the page. The '~' is needed because \\
    % only works if a paragraph has started.
    \includegraphics[scale=1]{univangers.jpg}~\\[1.5cm]

    \textsc{\LARGE Université d'Angers}\\[2cm]

   

    % Title
    \HRule \\[0.4cm]
    { \huge \bfseries Virtualisation d'un équipement d'une classe mobile}{\bfseries  \\[0.4cm] }

    \HRule \\[2cm]
    

    % Author and supervisor
    \begin{minipage}{0.4\textwidth}
      \begin{flushleft} \large
        CHERRUAU \textsc{Anthony}\\
        FRESNEAU \textsc{Quentin}
        POUPELIN \textsc{Bastien}\\
        THEBAUDIN \textsc{Corentin}
      \end{flushleft}
    \end{minipage}
    

    \vfill
    \HRule\\[2cm]
    % Bottom of the page
    {\large 30 Mars 2017}

  \end{center}
  \end{sffamily}
\end{titlepage}
\clearpage

\tableofcontents

\clearpage



\section{Présentation de l'existant}

\subsection{Contexte}
\paragraph{}
Les étudiants de la faculté des sciences de l’université d’Angers disposent d’un chariot de 40 tablettes (Samsung Galaxy Tab 4) qui permet aux étudiants, lors de travaux pratiques d’Android, d’observer leurs applications sur un support concret autre qu’un émulateur.\\


Lors des préparations ou aux termes des contrôles continus de cette unité, l’enseignant peut avoir besoin de déposer des fichiers, les récupérer, contrôler les répertoires, accéder aux logs et installer des applications sur les tablettes.
A cet effet, un projet d’étudiant de L3 informatique a été mené en 2015 afin de répondre au cahier des charges précédemment cité. Jérôme FOURMOND et Florentin NOEL ont donc développé une application pour le chariot, une application pc et une application android.\\

L’application chariot permet d’exécuter des scripts qui s’appuient sur sur le kit de développement (SDK) d’android, ADB, et qui possède les même fonctionnalités sauf qu’une marque concrète des événements produits sur la machine est enregistrée.  Elle s'exécute directement dans un terminal et  se comporte comme un serveur, elle reçoit des messages simples et exécute des les scripts associés à ces messages. L’application sert de liaison entre les tablettes et l’application PC.\\

L’application PC est une application graphique qui permet une gestion simple des tablettes connectées grâce à son interface. Les scripts sont facile à exécuter et le résultat des commandes exécutées sont affichés. \\

L’application android permet à l’utilisateur de la tablette d’obtenir des informations sur son appareil via une interface graphique claire et de commander un nettoyage de l'appareil lors de sa prochaine connexion au charriot.\\

Actuellement le chariot de tablette est composé de la sorte. Nous avons un pc principal sous pfSense qui est relié vers le wifi et l’ethernet. Ensuite nous avons un pc vertical sous xubuntu qui permet de voir les tablettes et d’utiliser les script pour des fonctionnalités sur les tablettes. Les tablettes sont reliés à deux hub usb un pour le chariot du bas et un autre pour le chariot du haut. Ces hub sont ensuite reliés au Xubuntu.\\

Le schéma ci dessous représente le chariot actuelle :\\

\includegraphics{representation_existant}
\clearpage
\subsection{Objectif du projet}

La finalité de ce projet est de permettre de n’avoir qu’un ordinateur contenant un hyperviseur de machines virtuelles, xenserver et deux machines virtuelle, le pfSense et le Xubuntu.
Cela permet de réduire le nombre de matériel requis pour le chariot. Au lieu d’avoir deux ordinateurs, tout est pilotables depuis un seul ordinateur et plus facile à gérer. De plus on peut prévoir de futures évolutions grâce à l’ajout de d’autres machines virtuelles au besoin.

Le schéma ci dessous représente le chariot désiré :\\

\includegraphics{representation_projet}

\clearpage
\section{Elaboration du projet}
\subsection{Partage du projet}

blabla

\subsection{Partage du travail(GANTT)}

Nous avons donc dû partager le travail. Pour cela nous avons donc défini des axes et avons réparti le travail. Pour cela nous avons un Gantt qui permet de voir ou l’on en ai et ce qu’il reste à faire.\\

image a inclure

\subsection{Matériel à disposition}

Au départ du projet plusieurs matériels ont été mis à notre disposition :
\begin{itemize}
\item Ordinateur portable HP EliteBook 8540w, i5M560, 4Go de ram
\item Un adaptateur HP usb/ethernet
\item Une clé usb 2.0 de 4Go
\item Une tablette Samsung galaxy tab
\item Une tablette Asus\\
\end{itemize}

Au fur et à mesure de l’avancement de notre projet nous avons rencontrés des problèmes matériels. C’est pour cela que de nouveaux matériels ont été mis à notre disposition :
\begin{itemize}
\item Ordinateur portable HP EliteBook 8560w, i7-2620M, 8Go
\item Un adaptateur Apple usb/ethernet
\item Un mini hub usb 3 port (brancher 2 tablettes)
\end{itemize}

\subsection{Réalisation du wiki}

L’une des consignes du projet était aussi de réaliser une page wikipédia contenant les étapes clés du projet, des tutoriels d’installation ou de résolution de problème et l’avancement de projet.
Notre tuteur nous a donc ouvert une section sur le wikipédia de l’université d’angers associant nos noms pour obtenir les droit d’écriture.\\

\url{http://wiki.info.univ-angers.fr/projets_etudiants:2017_l3pro_eg_virtualisation_classe_mobile}


\section{Mise en place de l'installation}
\subsection{XenServer}
\paragraph{Présentation\\}


Xen est un logiciel libre de virtualisation, ou plus précisément un hyperviseur de machine virtuelle basé sur centOS.\\

Xen permet d'exécuter plusieurs systèmes d'exploitation (et leurs applications) de manière isolée sur une même machine physique sur un grand nombre de plate-formes. Les systèmes d'exploitation invités partagent ainsi les ressources de la machine hôte.\\

Xen est un « paravirtualiseur » ou un « hyperviseur » de machines virtuelles. Les systèmes d'exploitation invités ont « conscience » du Xen sous-jacent, ils ont besoin d'être « portés » (adaptés) pour fonctionner sur Xen. Linux, NetBSD, FreeBSD, Plan 9 et GNU Hurd peuvent d'ores et déjà fonctionner sur Xen.\\

\paragraph{Dans le projet\\}

Pour débuter le projet, le choix de l’hyperviseur Xen à été fait en fonction des consignes notre tuteurs mais il aurait été possible d’utiliser d’autres logiciels libre gratuit équivalent tel que Proxmox. \\
Les premiers tests ont été réalisé avec la version 7.0 de XEN. L’installation se fait classiquement par le biais d’un ISO sur une clé USB bootable.

\includegraphics{xenserver18}\\

L’interface est plutôt sobre et ne permet de que quelques fonctionnalités simples comme le changement d’adresse IP ou le redémarrage de machines virtuelles.. Afin de pouvoir réaliser des opérations complexes comme la création d’une machine virtuelle, il faut utiliser les lignes de commandes ou un client externe dont nous parlerons dans la partie d de ce projet.\\

Les lignes de commande sont toutefois essentielles pour certaines fonctionnalités comme l’attribution d’un port USB à une machine virtuelle définit. Dans notre cas, la machine virtuelle Xubuntu devait pouvoir accéder à au moins 1 port USB pour être relié aux hub du chariot.

\subsection{PfSense}
\paragraph{Présentation\\}

PfSense est un routeur/pare-feu open source. Il utilise le pare-feu à états Packet Filter, des fonctions de routage et de NAT lui permettant de connecter plusieurs réseaux informatiques. Il comporte l'équivalent libre des outils et services utilisés habituellement sur des routeurs professionnels propriétaires. pfSense convient pour la sécurisation d'un réseau domestique ou de petite entreprise.


\paragraph{Dans le projet\\}

La machine virtuelle va nous permettre de faire le lien entre le wan et le lan. Le pfsense a deux interfaces réseaux. Une qui est la carte réseau de l’ordinateur et l’autre qui est l’adaptateur ethernet/usb. Le pfsense va permettre que xubuntu puisse avoir un lien vers l’extérieur et donc les tablette auront accès au wan. \\
PfSense utilise la version 2.3.2 qui au moment de l’installation la dernière version stable. Mais une nouvelle version est apparu entre temps la 2.3.3 mais nous n’avons pas mis à jour notre pfsense. Il y aura donc une évolution de Pfsense possible.

\subsection{Xubuntu}
\paragraph{Présentation\\}

Xubuntu est un système d'exploitation libre de type GNU/Linux. C'est un projet issu de la Fondation Ubuntu utilisant l'environnement de bureau graphique Xfce à la place d'Unity. Le projet Xubuntu est une distribution Linux dérivée de Ubuntu, car tous deux partagent exactement la même base, des logiciels communs (Synaptic), les mêmes dépôts APT, le même nom de code et le même cycle de développement.


\paragraph{Dans le projet\\}

Nous utilisons dans notre versions la Xubuntu 16.04, Xenial Xerus qui est la dernière version de Xubuntu. C’est une version 64 bits puisque que l’ordinateur prêté est un 64 bits. De plus les scripts développés par les étudiants sont adaptés pour une machine en 64 bits. Le choix de xubuntu est dû au fait que c’est un version plus légère de linux et cela permet sur une machine virtuelle d’économiser les ressources et d’avoir une machine virtuelle plus stable.
Le xubuntu va nous permettre de voir les tablettes et d’utiliser les script de gestion des tablettes. \\
Le choix de Xubuntu était imposé dès le début du projet car en prenant une Ubuntu par exemple, il y aurait eu  un problème sur le nombre de tablette détecté par ubuntu. La prise en charge de multiple matériel sur un usb est bloqué à sept sur Ubuntu d'où le choix de Xubuntu pour voir nos 40 tablettes. 
Pour le fonctionnement des scipt le kit de développement (SDK) d’android ADB a été ajouté à la machine.


\subsection{OpenXenManager}

Travaillant sous linux il nous fallait un logiciel pouvant se connecter à notre xenServer et avoir une interface graphique de nos machine virtuelle. Sous linux le seul logiciel est openXenManager qui est logiciel officiel pour un XenServer.

\includegraphics{openxenmanager}


\paragraph{exemple paragraphe}
\subparagraph{}
blabla

\paragraph{exemple paragraphe}
\subparagraph{}

\paragraph{exemple paragraphe}
\paragraph{}

\paragraph{exemple commande}
\begin{verbatim}
 ex commande
\end{verbatim}

\clearpage
\section{exemple partie}
\subsection{exemple sous partie}
\paragraph{}

exemple liste
\begin{itemize}
\item item 1
\item item 2 
\item item 3
\end{itemize}


\clearpage
\section{Références}

ex :
\url{www.google.fr}\




\end{document}